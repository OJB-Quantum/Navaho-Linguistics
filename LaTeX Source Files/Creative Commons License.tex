\large{This document is written by Onri Jay Benally, an Indigenous American born and raised on the Navaho tribe (Diné Bikeyah). Its purpose is to develop and collect Navaho terms associated with quantum information science and technology (QIST), highlighting on quantum hardware engineering. As you may find, this development is motivated by the drive not only to exploit linguistics as a way of better understanding emerging technology, but to also serve as a tool to contribute to the quantum community at large. With that, it may be possible to form a model around this effort in order to break down barriers to entry when it comes to approaching or incorporating oneself into the field of QIST. 
\\ 
\space
\\
\indent If you are interested in learning Navaho, the first section of this document will cover some basic ideas about the language. This will be followed by some examples of how one might think about names and labels from the Indigenous American perspective. It may be worth noting that although Navaho is considered to be a Tier 5 or Category 5 language in terms of difficulty for native English speakers, practicality is our main concern.
}


\begin{center}
\space
\end{center}


\begin{flushleft}
\title{\Large\textbf{Creative Commons License}}\\
\end{flushleft}

{
\large This work is licensed under the Creative Commons Attribution 4.0 International License. To view a copy of this license, visit \url{http://creativecommons.org/licenses/by/4.0/} or send a letter to Creative Commons, PO Box 1866, Mountain View, CA 94042, USA.
}

\begin{center}

\includegraphics{by.png}

\end{center}